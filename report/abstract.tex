\thispagestyle{plain}
\begin{center}
    \Large
    \textbf{Abstract}
\end{center}
\textit{This thesis about medical imaging aims to find alternative ways to map brain connectivity, utilizing the T1 MRI image instead of the diffusion MRI image, vastly improving the cost and time efficiency of the process. As a replacement for tractograpy, the currently used and accepted tool for processing the diffusion images, this thesis will reveal if there are any simple or complex relationship between radiomic features of the T1 image of the brain regarding the connected regions. It will also dive into fully convolutional approaches, in order to process the said T1 image. The results and conclusions are limited to the connectivity of the basal ganglia to other main cortical regions of the brain. It is in no way a generalized conclusion, but rather a proof of concept from experiments ran on a specific dataset, provided by Hospital Universitari de Bellvitge.}