\thispagestyle{plain}
\begin{center}
    \Large
    \textbf{Abstract}
\end{center}
\textit{This study investigates an alternative to diffusion MRI for mapping structural brain connectivity by predicting fractional anisotropy (FA) and mean diffusivity (MD) using radiomic features extracted from T1 and T2 anatomical MRI scans. The proposed method employs a feedforward neural network to synthesize FA and MD from radiomic data, aiming to improve the efficiency of data acquisition by eliminating the need for diffusion MRI to obtain connectivity metrics. The research focuses on the basal ganglia, a crucial region involved in brain structure and function that is significantly impacted by neurodegeneration in Huntington's disease. Including individuals with neurodegeneration makes it possible to assess the method's reliability and consistency. The approach shows promising results, achieving Pearson correlations of 85\% and 95\% for FA and MD, respectively. While prediction accuracy for FA decreases substantially, and MD moderately, when exclusively analyzing patients with Huntington's disease, mixing healthy controls with patients results in minimal decrease of performance.}