\thispagestyle{plain}
\begin{center}
    \Large
    \textbf{Abstract}
\end{center}
\textit{Anatomical magnetic resonance imaging (MRI), such as T1 and T2 weighted images, are commonly used to distinguish brain tissue types. Diffusion tensor imaging (DTI), a more specialized MRI technique, enables the mapping of white matter and the study of structural connectivity through metrics like fractional anisotropy (FA) and mean diffusivity (MD). However, DTI acquisition is time intensive and often excluded from standard clinical protocols.}\par
\textit{This study introduces a method to synthesize FA and MD images from widely available T1 and T2 weighted anatomical scans, reducing dependence on resource intensive DTI acquisition.}\par
\textit{Radiomics, a rapidly evolving field, focuses on extracting quantitative features from medical imaging, to identify biomarkers associated with clinical labels. It is a well established tool in oncology, particularly for tumor segmentation. Fully convolutional neural networks (FCNNs) are also employed for similar tasks, such as deriving clinical labels for tumor segmentation. However, their reliance on computationally intensive 3D convolutions often necessitates the use of 2D slices from 3D volumes, limiting their feasibility.}\par
\textit{We present a hybrid approach that combines the strengths of radiomics and neural networks. Specifically, we use a feedforward neural network (FNN) as a classification or regression head applied to radiomic features, preserving 3D spatial information during feature extraction while reducing the computational demands of 3D convolutions. This neural network head, akin to fully connected layers in traditional convolutional neural networks (CNNs), offers significant flexibility.}\par
\textit{Our experiments are limited to the basal ganglia, a crucial region involved in brain structure and function that is significantly impacted by neurodegeneration in Huntington’s disease (HD). The inclusion of HD patients allows for assessing the method's reliability and robustness. Additionally, we explore the potential of using the T1/T2 ratio as an input image, which has been proposed in recent studies, as a proxy for the myelin content of the brain. Since myelin plays a role in how DTI functions, the T1/T2 ratio may enhance model performance, effectively bridging anatomical MRI and DTI.}\par
\textit{The results indicate strong performance, with Pearson correlations of 85\% for FA and 95\% for MD predictions. While the correlation decreases significantly for FA and moderately for MD in HD patients, the model’s performance remains consistent when both healthy controls and patients are analyzed together. These findings underscore the promise of our hybrid approach for synthesizing structural connectivity images and improving accessibility to diffusion metrics.}