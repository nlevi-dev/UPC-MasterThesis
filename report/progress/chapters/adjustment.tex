
The main adjustment was changing the focus of the main goal of predicting relative connectivity, to predicting \ac{FA} and \ac{MD}. Due to the yielded results being underwhelming, regardless that with further fine tuning we could get the accuracy up to maybe around 0.7 which is much-much better than random guessing. And the \ac{FA} and \ac{MD} predictions look more promising and easier to work with, so we want to focus the remaining resources on fine tuning and experimenting with those models.\par

Other deviations from the original plans, is that we are not even going to try using a highly explainable model (such as decision trees), as it is evident at this point that the complexity of this problem is beyond the capabilities of said models. Furthermore we will not try fully convolutional approaches, after realizing that it does not fit the original hypothesis of 'can we predict relative connectivity from the radiomic features'. As well as other technical limitations, as a simple U-net with 3 downasmple and 3 upsample layers, were already huge enough so it would only work with a maximum batch size of 1, on a GPU with 24GBs of VRAM.\par

And lastly we would like to change the title of the thesis to 'Predicting Brain Connectivity Mapping Using Radiomics Features in Anatomical MRI' from 'Characterizing Neurodegeneration Patterns Using Radiomics' as it did not completely fit for what we are trying to do.